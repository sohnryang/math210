\documentclass{scrartcl}
\usepackage[margin=3cm]{geometry}
\usepackage{amsmath}
\usepackage{amssymb}
\usepackage{amsthm}
\usepackage{blindtext}
\usepackage{datetime}
\usepackage{fontspec}
\usepackage{graphicx}
\usepackage{kotex}
\usepackage{mathrsfs}
\usepackage{mathtools}
\usepackage{pgf,tikz,pgfplots}

\pgfplotsset{compat=1.15}
\usetikzlibrary{arrows}

\newcommand\Overline[2][0.4pt]{%
  \begin{tikzpicture}[baseline=(a.base)]
    \node[inner xsep=0pt,inner ysep=1.5pt] (a) {$#2$};
    \draw[line width= #1] (a.north west) -- (a.north east);
  \end{tikzpicture}
}
\newtheorem{theorem}{Theorem}

\setmainhangulfont{Noto Serif CJK KR}[
  UprightFont=* Light, BoldFont=* Bold,
  Script=Hangul, Language=Korean, AutoFakeSlant,
]
\setsanshangulfont{Noto Sans CJK KR}[
  UprightFont=* DemiLight, BoldFont=* Medium,
  Script=Hangul, Language=Korean
]
\setmathhangulfont{Noto Sans CJK KR}[
  SizeFeatures={
    {Size=-6,  Font=* Medium},
    {Size=6-9, Font=*},
    {Size=9-,  Font=* DemiLight},
  },
  Script=Hangul, Language=Korean
]
\title{MATH210: Homework 3 (due Mar. 21)}
\author{손량(20220323)}
\date{Last compiled on: \today, \currenttime}

\newcommand{\un}[1]{\ensuremath{\ \mathrm{#1}}}

\begin{document}
\maketitle

\section{Section 29 \#5}
Assume that \(f(x)\) is pure imaginary at each point \(x\) on the segment of real axis lying in \(D\).
Let \(F(z) = -\Overline{f(\Overline{z})}\).
The components of the functions can be written as
\[
  f(z) = u(x, y) + iv(x, y), \quad F(z) = U(x, y) + iV(x, y)
\]
where \(z = x + iy\).
By definition of \(F(z)\), we can write
\[
  F(z) = -\Overline{f(\Overline{z})} = -u(x, -y) + iv(x, -y)
\]
So \(U(x, y)\) and \(V(x, y)\) can be written as follows:
\begin{align}\label{sec1_comprel}
  U(x, y) = -u(x, t), \quad V(x, y) = v(x, t)
\end{align}
Where \(t = -y\).
Since \(f(x + it)\) is an analytic function of \(x + it\), the first order partial derivatives of the functions \(u(x, t)\) and \(v(x, t)\) are continuous throughout \(D\) and satisfies the Cauchy-Riemann equations.
\begin{align}\label{sec1_creq}
  u_x = v_t, \quad u_t = -v_x
\end{align}
Also, we can write
\begin{align*}
  U_x(x, y) &= -u_x(x, t), \quad V_y(x, y) = v_t(x, t) \frac{dt}{dy} = -v_t(x, t) \\
  U_y(x, y) &= -u_t(x, t) \frac{dt}{dy} = u_t(x, t), \quad V_x(x, y) = v_x(x, t)
\end{align*}
By (\ref{sec1_creq}), we can see that
\[
  U_x = V_y, \quad U_y = -V_x
\]
By the sufficient conditions of differentiability, \(F\) is differentiable throughout \(D\), so it is also analytic.
Moreover, since \(f(x)\) is pure imaginary on the segment of the imaginary axis lying in \(D\), \(u(x, 0) = 0\) holds on that segment.
Using (\ref{sec1_comprel}), we can write
\[
  F(x) = U(x, 0) + iV(x, 0) = -u(x, 0) + iv(x, 0) = iv(x, 0) = f(x)
\]
Thus, we can conclude that \(F(z) = f(z)\) at each point in the segment.
By the theorem in section~28, \(F(z) = f(z)\) holds throughout \(D\), so \(f(z) = -\Overline{f(\Overline{z})}\).

Let's prove the converse.
Assume that \(\Overline{f(z)} = -f(\Overline{z})\), then it can be written as:
\[
  u(x, y) - iv(x, y) = -u(x, -y) - iv(x, -y)
\]
If \((x, 0)\) is a point on the segment of the real axis lying on \(D\),
\[
  u(x, 0) - iv(x, 0) = -u(x, 0) - iv(x, 0)
\]
We see that \(u(x, 0) = 0\).
Hence \(f(x)\) is pure imaginary on the segment of the imaginary axis lying on \(D\).

\section{Section 30 \#9}
Assume that \(z = n\pi\) where \(n = 0, \pm 1, \pm 2, \dots\).
We can write
\begin{align*}
  \Overline{\exp(iz)} &= \Overline{\cos z + i\sin z} = \Overline{\cos (n\pi) + i\sin (n\pi)} = \cos (n\pi) - i\sin (n\pi) = \cos (n\pi) \\
                      &= \exp(iz) = \exp(i\Overline{z})
\end{align*}
and get the desired result.

Let's prove the converse.
Assume that \(\Overline{\exp(iz)} = \exp(i\Overline{z})\) holds.
The following holds for \(z = x + iy\).
\begin{align}
  \nonumber \Overline{\exp(iz)} &= \Overline{\exp(ix - y)} = \Overline{\exp(-y)\exp(ix)} = e^{-y} \Overline{\exp(ix)} = e^{-y} \Overline{\cos x + i\sin x} \\
                      \label{sec2_lhs}&= e^{-y} (\cos x - i\sin x) = e^{-y} (\cos (-x) + i\sin(-x)) = e^{-y} \exp(-ix)
\end{align}
We can also write
\begin{align}\label{sec2_rhs}
  \exp(i\Overline{z}) = \exp{i(x - iy)} = e^y \exp(ix)
\end{align}
From (\ref{sec2_lhs}) and (\ref{sec2_rhs}), \(\Overline{\exp(iz)} \exp(i\Overline{z}) = 1\) holds.
For \(\Overline{\exp(iz)} = \exp(i\Overline{z})\) to hold, from
\[
  |\Overline{\exp(iz)} \exp(i\Overline{z})| = 1
\]
\(|\Overline{\exp(iz)}| = |\exp(i\Overline{z})| = 1\) should hold.
We can write
\[
  |\Overline{\exp(iz)}| = |e^{-y} \exp(-ix)| = e^{-y} = 1,\, |\exp(i\Overline{z})| = |e^y \exp(ix)| = e^y = 1
\]
So we know that \(y = 0\).
Then, \(\Overline{\exp(iz)} = \exp(i\Overline{z})\) so \(\exp(-ix) = \exp(ix)\).
We can write
\[
  \exp(-ix) = \cos (-x) + i\sin(-x) = \cos x - i\sin x, \quad \exp(ix) = \cos x + i\sin x
\]
\(\sin x = 0\) is satisfied for \(x = 0, \pm \pi, \pm 2\pi, \dots\), so the converse is also true.

\section{Section 30 \#13}
Since \(f\) is analytic in \(D\), \(u\) and \(v\) have continuous partial derivatives in \(D\) and they satisfy the Cauchy-Riemann equations.
\begin{align}\label{sec3_creq}
  u_x = v_y, \quad u_y = -v_x
\end{align}
Using rules for differentiation, we get
\begin{align*}
  U_x(x, y) &= u_x(x, y) e^{u(x, y)} \cos v(x, y) - v_x(x, y) e^{u(x, y)} \sin v(x, y) \\
  U_y(x, y) &= u_y(x, y) e^{u(x, y)} \cos v(x, y) - v_y(x, y) e^{u(x, y)} \sin v(x, y) \\
  V_x(x, y) &= u_x(x, y) e^{u(x, y)} \sin v(x, y) + v_x(x, y) e^{u(x, y)} \cos v(x, y) \\
  V_y(x, y) &= u_y(x, y) e^{u(x, y)} \sin v(x, y) + v_y(x, y) e^{u(x, y)} \cos v(x, y)
\end{align*}
Using (\ref{sec3_creq}), we can write
\begin{align*}
  U_x(x, y) &= u_x(x, y) e^{u(x, y)} \cos v(x, y) + u_y(x, y) e^{u(x, y)} \sin v(x, y) \\
  U_y(x, y) &= u_y(x, y) e^{u(x, y)} \cos v(x, y) - u_x(x, y) e^{u(x, y)} \sin v(x, y) \\
  V_x(x, y) &= u_x(x, y) e^{u(x, y)} \sin v(x, y) - u_y(x, y) e^{u(x, y)} \cos v(x, y) \\
  V_y(x, y) &= u_y(x, y) e^{u(x, y)} \sin v(x, y) + u_x(x, y) e^{u(x, y)} \cos v(x, y)
\end{align*}
From this, we know that
\[
  U_x = V_y, \quad U_y = -V_x
\]
Let \(F(z) = U(x, y) + iV(x, y)\).
Then \(U\) and \(V\) have continuous partial derivatives in \(D\), and they satisfy the Cauchy-Riemann equations.
By the sufficient condition for differentiability, \(F\) is an analytic function.
Using the theorem in section~27, the component functions \(U\) and \(V\) are harmonic in \(D\).

\section{Section 30 \#14}
\subsection{Solution for (a)}
For any nonzero \(x \in \mathbb{C}\), \(x^1 = x\) and we know that \(x^0 = x^{1 - 1} = x / x = 1\).

Then \((e^z)^0 = e^{0 \cdot z} = 1\) since \(e \not = 0\) and \(e^z \not = 0\).

Suppose that \((e^z)^k = e^{kz}\) for \(k \geq 0\). Using the rules of exponentation,
\[
  (e^z)^{k + 1} = (e^z)^k (e^z)^1 = e^{kz} e^z = e^{(k + 1)z}
\]
By mathematical induction, we can see that \((e^z)^n = e^{nz}\) holds for all nonnegative integer \(n\).

\subsection{Solution for (b)}
For \(m = -n = 1, 2, \dots\), \((e^z)^n\) = \((1 / e^z)^m\) holds by the definition of negative integer in section~8.
By the result proved in (a), \((1 / e^z)^m = 1 / (e^z)^m = 1/ e^{mz}\) holds.
From property~6 in section~30, it follows that \(1 / e^{mz} = e^{-mz} = e^{nz}\).
So, \((e^z)^n = e^{nz}\) also holds for negative integer \(n\).

\section{Section 33 \#2 (c)}
If \(-1 + \sqrt{3}i\) then \(r = 2\) and \(\Theta = 2\pi / 3\).
Hence
\[
  \log (-1 + \sqrt{3}i) = \ln r + i(\Theta + 2n\pi) = \ln 2 + i \left( \frac{2\pi}{3} + 2n\pi \right) = \ln 2 + 2 \left( n + \frac{1}{3} \right) \pi i
\]
where \(n = 0, \pm 1, \pm 2, \dots\).

\section{Section 33 \#7}
Let \(z = x + yi = re^{i\theta}\).
As written in section~7, \(r = |z| = \sqrt{x^2 + y^2}\) and \(\tan \theta = y / x\) holds, so \(\theta = \tan^{-1} (y / x)\) satisfies \(x + yi = re^{i\theta}\).
Using the definition in section~33, we can write
\begin{align*}
  \log z &= \ln r + i\theta \quad (r > 0,\, \alpha < \theta < \alpha + 2\pi) \\
         &= \ln \sqrt{x^2 + y^2} + i\tan^{-1} \left( \frac{y}{x} \right) \\
         &= \frac{1}{2} \ln (x^2 + y^2) + i\tan^{-1} \left( \frac{y}{x} \right)
\end{align*}
We can write the component functions as follows:
\[
  u(x, y) = \frac{1}{2} \ln (x^2 + y^2), \quad v(x, y) = \tan^{-1} \left( \frac{y}{x} \right)
\]
\(u\) and \(v\) has continuous partial derivatives in its domain, so we can differentiate them.
\begin{align*}
  u_x(x, y) &= \frac{x}{x^2 + y^2}, \quad u_y(x, y) = \frac{y}{x^2 + y^2} \\
  v_x(x, y) &= -\frac{y}{x^2} \frac{1}{1 + \left( \frac{y}{x} \right)^2} = -\frac{y}{x^2 + y^2} \\
  v_y(x, y) &= \frac{1}{x} \frac{1}{1 + \left( \frac{y}{x} \right)^2} = \frac{x}{x^2 + y^2}
\end{align*}
Then \(u_x = v_y, u_y = -v_x\) so the Cauchy-Riemann equation holds, and \(\log z\) is analytic in its domain by the sufficient conditions of differentiability.
Thus, \(\log z\) is differentiable in its domain and we can write
\[
  \frac{d}{dz} \log z = u_x(x, y) + iv_x(x, y) = \frac{x}{x^2 + y^2} - \frac{iy}{x^2 + y^2} = \frac{x - yi}{x^2 + y^2} = \frac{\Overline{z}}{|z|^2} = \frac{\Overline{z}}{z\Overline{z}} = \frac{1}{z}
\]
and get the desired result.

\section{Section 33 \#8}
By the definition in section~31, we can write
\[
  \log z = \ln |z| + i \arg z
\]
Since \(\log z = i\pi / 2\), it is clear that \(|z| = 1\) and \(\arg z = \pi / 2 + 2n\pi\) where \(n = 0, \pm 1, \pm 2, \dots\). Then \(z\) can be written as
\[
  z = 1 \cdot \exp \left( i \left( \frac{\pi}{2} + 2n\pi \right) \right) = e^{i\pi / 2} = \cos \left( \frac{\pi}{2} \right) + i\sin \left( \frac{\pi}{2} \right) = i
\]
and get the root.

\end{document}
