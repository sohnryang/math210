\documentclass{scrartcl}
\usepackage[margin=3cm]{geometry}
\usepackage{amsmath}
\usepackage{amssymb}
\usepackage{amsthm}
\usepackage{blindtext}
\usepackage{datetime}
\usepackage{fontspec}
\usepackage{graphicx}
\usepackage{kotex}
\usepackage{mathrsfs}
\usepackage{mathtools}
\usepackage{pgf,tikz,pgfplots}

\usepackage[headsepline]{scrlayer-scrpage}

\pgfplotsset{compat=1.15}
\usetikzlibrary{arrows}
\newtheorem{theorem}{Theorem}

\newcommand\Overline[2][0.8pt]{%
  \begin{tikzpicture}[baseline=(a.base)]
    \node[inner xsep=0pt,inner ysep=1.5pt] (a) {$#2$};
    \draw[line width= #1] (a.north west) -- (a.north east);
  \end{tikzpicture}
}

\setmainhangulfont{Noto Serif CJK KR}[
  UprightFont=* Light, BoldFont=* Bold,
  Script=Hangul, Language=Korean, AutoFakeSlant,
]
\setsanshangulfont{Noto Sans CJK KR}[
  UprightFont=* DemiLight, BoldFont=* Medium,
  Script=Hangul, Language=Korean
]
\setmathhangulfont{Noto Sans CJK KR}[
  SizeFeatures={
    {Size=-6,  Font=* Medium},
    {Size=6-9, Font=*},
    {Size=9-,  Font=* DemiLight},
  },
  Script=Hangul, Language=Korean
]
\title{Homework 1 (due Mar. 7)}
\author{손량(20220323)}
\date{Last compiled on: \today, \currenttime}

\newcommand{\un}[1]{\ensuremath{\ \mathrm{#1}}}

\begin{document}
\maketitle

\section{Section 20 \#4}
By definition, \(f'(z_0), g'(z_0)\) can be written as follows:

\[
  f'(z_0) = \lim_{z \to z_0} \frac{f(z) - f(z_0)}{z - z_0},\, g'(z_0) = \lim_{z \to z_0} \frac{g(z) - g(z_0)}{z - z_0}
\]

By properties of limits,

\[
  \frac{f'(z_0)}{g'(z_0)} = \frac{\lim_{z \to z_0} \frac{f(z) - f(z_0)}{z - z_0}}{\lim_{z \to z_0} \frac{g(z) - g(z_0)}{z - z_0}} = \lim_{z \to z_0} \frac{\frac{f(z) - f(z_0)}{z - z_0}}{\frac{g(z) - g(z_0)}{z - z_0}} = \lim_{z \to z_0} \frac{f(z)}{g(z)}
\]

\section{Section 20 \#9}
\(\Delta w\) can be written as follows, if \(z = 0\).

\[
  \Delta w = f(z + \Delta z) - f(z) = f(\Delta z) - f(0) = f(\Delta z) = \frac{\Overline{\Delta z}^2}{\Delta z}
\]

Then \(\Delta w/\Delta z = \left(\Overline{\Delta z} / \Delta z\right)^2\).
If \(\Delta z\) is on the real axis, \(\Overline{\Delta z} = \Delta z\) and \(\left(\Overline{\Delta z} / \Delta z\right)^2 = \left( \Delta z / \Delta z \right)^2 = 1\).
If \(\Delta z\) is on the imaginary axis, \(\Overline{\Delta z} = -\Delta z\) and \(\Delta w / \Delta z = \left( \Overline{\Delta z} / \Delta z \right)^2 = \left( -\Delta z / \Delta z \right)^2 = 1\).
However, on each nonzero point on line \(\Delta y = \Delta x\) where \(\Delta z\) is written as \(\Delta x + i\Delta y\),

\[
  \frac{\Delta w}{\Delta z} = \left( \frac{\Overline{\Delta z}}{\Delta z} \right)^2 = \left( \frac{\Delta x - i\Delta x}{\Delta x + i\Delta x} \right)^2 = \left( \frac{1 - i}{1 + i} \right)^2 = (-i)^2 = -1
\]

If the limit of \(\Delta w / \Delta z\) exists, it can be found by letting \(\Delta z\) approach the origin in any manner.
However, the limit is 1 if \(\Delta z\) approached the origin along the real or imaginary axis, or \(-1\) if \(\Delta z\) approached the origin along the line \(\Delta y = \Delta x\).
Since limits are unique, it is a contradiction and \(f'(0)\) does not exist.

\section{Section 24 \#3 (c)}
\(f(z)\) can be written as follows:

\[
  f(z) = f(x + iy) = (x + iy)y = xy + iy^2
\]

Let \(u(x, y) := xy\) and \(v(x, y) = y^2\), then \(f(z) = u(x, y) + iv(x, y)\).
If \(f\) is differentible at \(z_0 = x_0 + iy_0\), the first-order partial derivatives of \(u\) and \(v\) must exist at \((x_0, y_0)\), and Cauchy-Riemann equation must be satisfied.
Since the partial derivatives exist at every \((x_0, y_0)\) since \(u\) and \(v\) are polynomial functions, let's check the Cauchy-Riemann equation.

\begin{align*}
  u_x(x_0, y_0) = v_y(x_0, y_0) &\Longleftrightarrow y_0 = 2y_0 \\
  u_y(x_0, y_0) = -v_x(x_0, y_0) &\Longleftrightarrow x_0 = 0
\end{align*}

Only \(x_0 = 0, y_0 = 0\) satistifes the equation.
\(f\) cannot be differentible at points other than \(0\).
Let's check if it is differentible at \(0\):

\begin{align*}
  \lim_{\Delta z \to 0} \frac{f(\Delta z) - f(0)}{\Delta z - 0} =& \lim_{\Delta z \to 0} \frac{f(\Delta z)}{\Delta z} = \lim_{\Delta z \to 0} \frac{\Delta z \operatorname{Im} \Delta z}{\Delta z} \\
  =& \lim_{\Delta z \to 0} \operatorname{Im} \Delta z = 0
\end{align*}

\(f\) is only differentible at \(0\), and \(f'(0) = 0\).

\section{Section 24 \#7}
\subsection{Solution for (a)}
According to problem \#6, the following holds for \(f'(z_0)\) where \(z_0 = r_0\exp(i\theta_0)\) and \(f\) is differentible at \(z_0\): (\(u_r\) and \(v_r\) are evaluated at \((r_0, \theta_0)\))

\[
  f'(z_0) = u_x + iv_x = e^{-i\theta_0} (u_r + iv_r)
\]

By the polar form of Cauchy-Riemann equation, \(u_r\) and \(v_r\) can be written as follows (\(u_\theta\) and \(v_\theta\) are evaluated at \((r_0, \theta_0)\))

\[
  u_r = v_\theta / r, \quad v_r = -u_\theta / r
\]

Substitution gives us the result:

\[
  f'(z_0) = \frac{1}{r_0 e^{i\theta_0}} (v_\theta - iu_\theta) = \frac{-i}{z_0} (u_\theta + iv_\theta)
\]

\subsection{Solution for (b)}
\(f(z)\) can be written as follows, where \(z = r\exp(i\theta) \not = 0\):

\[
  f(z) = \frac{1}{z} = \frac{1}{r e^{i\theta}} = \frac{1}{r} (\cos (-\theta) + i\sin (-\theta)) = \frac{1}{r} (\cos \theta - i\sin \theta)
\]

The component functions are

\[
  u(r, \theta) = \frac{\cos \theta}{r}, \quad v(r, \theta) = \frac{-\sin \theta}{r}
\]

Since \(f(z)\) is differentible in \(\mathbb{C} \setminus \{0\}\), we can use the relation in (a).
Plugging those to the expression derived in (a), we get the desired result.

\begin{align*}
  f'(z) &= \frac{-i}{z} (u_\theta + iv_\theta) = \frac{-i}{z} \left(\frac{-\sin \theta}{r} + i \frac{-\cos \theta}{r}\right) \\
  &= -\frac{i}{zr} \left( \sin(-\theta) - i\cos(-\theta) \right) = -\frac{1}{zr} (\cos(-\theta) + i\sin(-\theta)) \\
  &= -\frac{1}{zre^{i\theta}} = -\frac{1}{z^2}
\end{align*}

\section{Section 24 \#8}
\subsection{Solution for (a)}
Using the chain rule,

\begin{align*}
  \frac{\partial F}{\partial \Overline{z}} &= \frac{\partial F}{\partial x} \frac{\partial x}{\partial \Overline{z}} + \frac{\partial F}{\partial y} \frac{\partial y}{\partial \Overline{z}} = \frac{\partial F}{\partial x} \left( \frac{\partial}{\partial \Overline{z}} \frac{z + \Overline{z}}{2} \right) + \frac{\partial F}{\partial y} \left( \frac{\partial}{\partial \Overline{z}} \frac{z - \Overline{z}}{2i} \right) \\
 &= \frac{1}{2} \frac{\partial F}{\partial x} - \frac{1}{2i} \frac{\partial F}{\partial y} = \frac{1}{2} \left( \frac{\partial F}{\partial x} + i \frac{\partial F}{\partial y} \right)
\end{align*}

\subsection{Solution for (b)}
Using the defined operator,

\begin{align*}
  \frac{\partial f}{\partial \Overline{z}} = \frac{1}{2}\left( \frac{\partial f}{\partial x} + i \frac{\partial f}{\partial y} \right) = \frac{1}{2}\left[ (u_x + iv_x) + i(u_y + iv_y) \right] = \frac{1}{2}\left[ (u_x - v_y) + i(v_x + u_y) \right]
\end{align*}

Since the first order derivatives of the real and imaginary components of \(f\) satisfy the Cauchy-Riemann equations,

\[
  u_x = v_y, \quad u_y = -v_x
\]

Both \(u_x - v_y\) and \(v_x + u_y\) are zero, giving us the desired result, \(\partial f / \partial \Overline{z} = 0\).

\end{document}
