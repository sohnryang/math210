\documentclass{scrartcl}
\usepackage[margin=3cm]{geometry}
\usepackage{amsmath}
\usepackage{amssymb}
\usepackage{amsthm}
\usepackage{blindtext}
\usepackage{datetime}
\usepackage{fontspec}
\usepackage{graphicx}
\usepackage{kotex}
\usepackage{mathrsfs}
\usepackage{mathtools}
\usepackage{pgf,tikz,pgfplots}

\pgfplotsset{compat=1.15}
\usetikzlibrary{arrows}

\newcommand\Overline[2][0.8pt]{%
  \begin{tikzpicture}[baseline=(a.base)]
    \node[inner xsep=0pt,inner ysep=1.5pt] (a) {$#2$};
    \draw[line width= #1] (a.north west) -- (a.north east);
  \end{tikzpicture}
}
\newtheorem{theorem}{Theorem}

\setmainhangulfont{Noto Serif CJK KR}[
  UprightFont=* Light, BoldFont=* Bold,
  Script=Hangul, Language=Korean, AutoFakeSlant,
]
\setsanshangulfont{Noto Sans CJK KR}[
  UprightFont=* DemiLight, BoldFont=* Medium,
  Script=Hangul, Language=Korean
]
\setmathhangulfont{Noto Sans CJK KR}[
  SizeFeatures={
    {Size=-6,  Font=* Medium},
    {Size=6-9, Font=*},
    {Size=9-,  Font=* DemiLight},
  },
  Script=Hangul, Language=Korean
]
\title{MATH210: Homework 10 (due May. 16)}
\author{손량(20220323)}
\date{Last compiled on: \today, \currenttime}

\newcommand{\un}[1]{\ensuremath{\ \mathrm{#1}}}
\newcommand{\imag}{\operatorname{Im}}
\newcommand{\real}{\operatorname{Re}}
\newcommand{\Log}{\operatorname{Log}}
\newcommand{\Arg}{\operatorname{Arg}}
\DeclareMathOperator*{\Res}{Res}

\begin{document}
\maketitle

\section{Section 83 \#3 (b)}
Let \(p(z) = \exp(zt)\), \(q(z) = \sinh z\) and \(f(z) = p(z) / q(z)\).
Since \(p(\pi i) \not = 0\), \(q(\pi i) = 0\) and \(q'(\pi i) \not = 0\), we can apply theorem~2 in section~83 so
\begin{align*}
  \Res_{z = \pi i} f(z)
  = \frac{p(\pi i)}{q'(\pi i)}
  = \frac{\exp(\pi it)}{\cosh(\pi i)}
  = -e^{\pi it}
  = -\cos (\pi t) - i\sin(\pi t)
\end{align*}
Also, \(p(-\pi i) \not = 0\), \(q(-\pi i) = 0\) and \(q'(-\pi i) \not = 0\) we can apply theorem~2 in section~83 again,
\begin{align*}
  \Res_{z = -\pi i} f(z)
  = \frac{p(-\pi i)}{q'(-\pi i)}
  = \frac{\exp(-\pi it)}{\cosh(-\pi i)}
  = -e^{\pi it}
  = -\cos (\pi t) + i\sin (\pi t)
\end{align*}
and get the desired result.
\begin{align*}
  \Res_{z = \pi i} f(z) + \Res_{z = -\pi i} f(z) = -2\cos (\pi t)
\end{align*}

\section{Section 83 \#4 (b)}
Let \(p(z) = \sinh z,\, q(z) = \cosh z\) and \(f(z) = \tanh z = p(z) / q(z)\).
Since \(p(z_n) \not = 0,\, q(z_n) = 0\) and \(q(z_n) \not = 0 \quad (n = 0, \pm 1, \pm 2, \dots)\), we can apply theorem~2 in section~83 so
\begin{align*}
  \Res_{z = z_n} f(z)
  = \frac{p(z_n)}{q'(z_n)}
  = \frac{\sinh z_n}{\sinh z_n}
  = 1 \quad (n = 0, \pm 1, \pm 2, \dots)
\end{align*}
and get the desired result.

\section{Section 83 \#5 (a)}
Let \(p(z) = \sin z,\, q(z) = \cos z\) and \(f(z) = \tan z = p(z) / q(z)\) where \(z \not = (n + 1 / 2) \pi \quad (n = 0, \pm 1, \pm 2, \dots)\).
Since \(q(z) = 0\) where \(z = z_n := (n + 1/2)\pi \quad (n = 0, \pm 1, \pm 2, \dots)\), \(f(z)\) has isolated singularities at \(z_n \quad (n = 0, \pm 1, \pm 2, \dots)\).
Only \(-\pi / 2, \pi / 2\) are isolated singularities inside \(C\), so by using Cauchy's residue theorem,
\begin{align*}
  \int_C f(z) dz = 2\pi i \left( \Res_{z = -\pi / 2} f(z) + \Res_{z = \pi / 2} f(z) \right)
\end{align*}
Since \(p(-\pi / 2) \not = 0,\, p(\pi / 2) \not = 0\, q(-\pi / 2) = q(\pi / 2) = 0,\, q'(-\pi / 2) \not = 0\) and \(q'(\pi / 2) \not = 0\), we can apply theorem~2 in section~83.
\begin{align*}
  \Res_{z = -\pi / 2} f(z)
  &= \frac{p(-\pi / 2)}{q'(-\pi / 2)}
  = \frac{\sin (-\pi / 2)}{-\sin (-\pi / 2)} = -1 \\
  \Res_{z = \pi / 2} f(z)
  &= \frac{p(\pi / 2)}{q'(\pi / 2)}
  = \frac{\sin (\pi / 2)}{-\sin (\pi / 2)} = -1
\end{align*}
Then
\begin{align*}
  \int_C \tan z dz
  = 2\pi i \left( \Res_{z = -\pi / 2} f(z) + \Res_{z = \pi / 2} f(z) \right)
  = -4\pi i
\end{align*}

\section{Section 83 \#6}
Let \(q(z) = z^2 \sin z\) and \(f(z) = 1 / q(z)\).
Since \(q(z)\) is analytic and \(q(n\pi) = 0\) where \(n = 0, \pm 1, \pm 2, \dots\), \(f(z)\) has isolated singularities at \(0, \pm \pi, \pm 2\pi, \dots\).
The isolated singularities inside \(C_N\) are \(-N\pi, -(N - 1)\pi, \dots, -\pi, 0, \pi, \dots, (N - 1)\pi, N\pi\).
By Cauchy's residue theorem, we can write
\begin{align}\label{sec4_residue_sum}
  \int_{C_N} f(z) dz
  = 2\pi i \sum^N_{n = -N} \Res_{z = n\pi} f(z)
\end{align}
For \(n\pi\) where \(n \not = 0\), \(q(n\pi) = 0\) and \(q'(n\pi) \not = 0\) so by applying theorem~2 in section~83,
\begin{align*}
  \Res_{z = n\pi} f(z) = \frac{1}{q'(n\pi)} = \frac{1}{2n\pi \sin (n\pi) + (n\pi)^2 \cos (n\pi)} = \frac{(-1)^n}{n^2 \pi^2}
\end{align*}
From the Maclaurin series expansion of \(\sin z\),
\begin{align*}
  q(z)
  &= z^2 \sum^\infty_{n = 0} (-1)^n \frac{z^{2n + 1}}{(2n + 1)!}
  = \sum^\infty_{n = 0} (-1)^n \frac{z^{2n + 3}}{(2n + 1)!} \\
  &= z^3 - \frac{z^5}{3!} + \frac{z^7}{5!} - \cdots \quad (|z| < \infty)
\end{align*}
\(q(z)\) has zero of order 3 at 0.
By theorem~1 in section~83, \(f(z)\) has a pole of order 3 at 0.
Applying the theorem in section~80, \(f(z)\) can be written as \(\phi(z) / z^3\) where \(\phi(z)\) is analytic and nonzero at 0.
For \(0 < |z| < \pi\), \(\phi(z)\) can be written as \(z / \sin z\) so for \(0 < |z| < \pi\),
\begin{align}
  \nonumber\phi'(z) &= \frac{\sin z - z\cos z}{\sin^2 z} = (1 - z\cot z) \csc z \\
  \label{sec4_limit}\phi''(0) &= \lim_{h \to 0} \frac{\phi'(h) - \phi'(0)}{h}
\end{align}
Since \(\phi'(z)\) is continuous function, \(\lim_{z \to 0} \phi'(z) = \phi'(0)\) so by L'Hôspital's rule,
\begin{align*}
  \lim_{z \to 0} \frac{\sin z - z \cos z}{\sin^2 z}
  = \lim_{z \to 0} \frac{z\sin z}{2\cos z\sin z}
  = \lim_{z \to 0} \frac{z}{2\cos z}
  = 0
\end{align*}
Then the limit in (\ref{sec4_limit}) can be caculated as
\begin{align*}
  \phi''(0)
  &= \lim_{h \to 0} \frac{\phi'(h) - \phi'(0)}{h}
  = \lim_{h \to 0} \frac{\phi'(h)}{h}
  = \lim_{h \to 0} \frac{\sin h - h\cos h}{h\sin^2 h} \\
  &= \lim_{h \to 0} \frac{h \sin h}{\sin^2 h + 2h\cos h\sin h}
  = \lim_{h \to 0} \frac{1}{\frac{\sin h}{h} + 2\cos h}
  = \frac{1}{3}
\end{align*}
Applying the theorem in section~80,
\begin{align*}
  \Res_{z = 0} f(z) = \frac{\phi''(0)}{2!} = \frac{1}{6}
\end{align*}
and (\ref{sec4_residue_sum}) can be written as
\begin{align*}
  \int_{C_N} f(z) dz
  &= 2\pi i \left[ \frac{1}{6} + \sum^N_{n = -N, n \not = 0} \frac{(-1)^n}{n^2 \pi^2} \right] \\
  &= 2\pi i \left[ \frac{1}{6} + 2\sum^N_{n = 1} \frac{(-1)^n}{n^2 \pi^2} \right]
\end{align*}

From \(\sin z = \sin (x + iy) = \sin x \cosh y + i\cos x \sinh y\),
\begin{align*}
  |\sin z|^2 = (\sin x \cosh y)^2 + (\cos x \sinh y)^2 = \sin^2 x + \sinh^2 y
\end{align*}
so \(|\sin z| \geq |\sin x \cosh y| \geq |\sin x|\) as \(\cosh y \geq 1\) for all real \(y\) and \(|\sin z|^2 \geq \sinh^2 y\) as \(\sin^2 x \geq 0\) for all real \(x\).
Then for all \(z\) on contour \(C_N\), one of \(|\sin z| \geq |\sin x| = |\sin (\pm(N + 1/2)\pi)| = 1\) and \(|\sin z| \geq |\sinh y| = |\sinh (3\pi / 2)| > 1\) holds.
Thus, \(|\sin z| \geq 1\) so
\begin{align*}
  |f(z)| = \left| \frac{1}{z^2 \sin z} \right| \leq \left| \frac{1}{z^2} \right| \leq \frac{1}{[(N + 1/2)\pi]^2}
\end{align*}
and we can write
\begin{align*}
  \int_{C_N} f(z) dz \leq \frac{1}{[(N + 1/2)\pi]^2} \cdot 8 \left( N + \frac{1}{2} \right)\pi \leq \frac{16}{(2N + 1)\pi}
\end{align*}
So the integral on the left hand side tends to zero as \(N \to \infty\).
Thus, it follows that
\begin{align*}
  2\pi i \left[ \frac{1}{6} + 2\sum^\infty_{n = 1} \frac{(-1)^n}{n^2 \pi^2} \right] = 0
\end{align*}
so
\begin{align*}
  \sum^\infty_{n = 1} \frac{(-1)^n}{n^2 \pi^2} = -\frac{1}{12}
\end{align*}
and get the desired result.
\begin{align*}
  \sum^\infty_{n = 1} \frac{(-1)^{n + 1}}{n^2} = \frac{\pi^2}{12}
\end{align*}

\section{Section 86 \#4}
Let \(p(z) = x^2,\, q(z) = x^6 + 1\) and \(f(z) = x^2 / (x^6 + 1) = p(z) / q(z)\).
\(f(z)\) has isolated singularities at the zeros of \(q(z)\), which are sixth roots of -1 and \(f(z)\) is analytic anywhere else.
The complex roots of -1 are \(c_k = \exp(i(\pi + 2k\pi) / 6) \quad (k = 0, 1, \dots, 6)\) and none of them lies on the real axis.
\(c_0, c_1, c_2\) lie in the upper half plane, so they are inside a semicircular region bounded by \(z = x \quad (x \in [-R, R])\) and upper half \(C_R\) of the circle \(|z| = R\), where \(R > 1\).
By Cauchy's residue theorem,
\begin{align*}
  \int^R_{-R} f(x) dx + \int_{C_R} f(z) dz
  = 2\pi i \left( \Res_{z = c_0} f(z) + \Res_{z = c_1} f(z) + \Res_{z = c_2} f(z) \right)
\end{align*}
We can apply theorem~2 in section~83.
\begin{align*}
  \Res_{z = c_k} f(z)
  = \frac{p(c_k)}{q'(c_k)}
  = \frac{c_k^2}{6c_k^5}
  = \frac{c_k^3}{6c_k^6}
  = -\frac{c_k^3}{6}
\end{align*}
Thus,
\begin{align*}
  \int^R_{-R} f(x) dx + \int_{C_R} f(z) dz
  &= -\frac{\pi i}{3} (c_0^3 + c_1^3 + c_2^3) \\
  &= -\frac{\pi i}{3} (e^{i\pi / 2} + e^{3i\pi / 2} + e^{5i\pi / 2}) \\
  &= -\frac{\pi i}{3} (i - i + i) = \frac{\pi}{3}
\end{align*}
For \(z\) on \(C_R\), we can write
\begin{align*}
  \left| \frac{z^2}{z^6 + 1} \right| = \frac{R^2}{|z^6 + 1|} \leq \frac{R^2}{|z^6| - 1} = \frac{R^2}{R^6 - 1}
\end{align*}
so
\begin{align*}
  \left| \int_{C_R} f(z) dz \right| \leq \frac{R^2}{R^6 - 1} \cdot \pi R = \frac{\pi R^3}{R^6 - 1}
\end{align*}
and the contour integral over \(C_R\) tends to zero as \(R \to \infty\).
Since \(f(-z) = f(z)\) for all \(z\), we can write
\begin{align*}
  \int^\infty_0 \frac{x^2 dx}{x^6 + 1}
  = \lim_{R \to \infty} \int^R_0 \frac{x^2 dx}{x^6 + 1}
  = \lim_{R \to \infty} \frac{1}{2} \left( \frac{\pi}{3} - \int_{C_R} f(z) dz \right)
  = \frac{\pi}{6}
\end{align*}

\section{Section 86 \#6}
Let \(p(z) = x^2,\, q(z) = (x^2 + 9)(x^2 + 4)^2\) and \(f(z) = p(z) / q(z)\).
\(f(z)\) has isolated singularities at the zeros of \(q(z)\), which are \(\pm 3i, \pm 2i\) and \(f(z)\) is analytic anywhere else.
None of the isolated singularities lies on the real axis.
\(3i\) and \(2i\) lie in the upper half plane, so they are inside a semicircular region bounded by \(z = x \quad (x \in [-R, R])\) and upper half \(C_R\) of the circle \(|z| = R\), where \(R > 3\).
By Cauchy's residue theorem,
\begin{align*}
  \int^R_{-R} f(x) dx + \int_{C_R} f(z) dz
  = 2\pi i \left( \Res_{z = 3i} f(z) + \Res_{z = 2i} f(z) \right)
\end{align*}
Since \(q(3i) = 0\) and \(q'(3i) \not = 0\), we can apply theorem~2 in section~83.
\begin{align*}
  \Res_{z = 3i} f(z) = \frac{p(3i)}{q'(3i)} = \frac{-9}{6i(-9 + 4)^2} = \frac{3i}{50}
\end{align*}
Since \(q(2i) = q'(2i) = 0\) and \(q''(2i) \not = 0\), \(2i\) is a zero of order 2 of \(q\), so \(2i\) is a pole of order 2 of \(f\) by theorem~1 in section~83.
Applying the theorem in section~80, \(f(z)\) can be written as \(\phi(z) / (z - 2i)^2\) where \(\phi(z)\) is analytic and nonzero at \(2i\).
For \(0 < |z - 2i| < \infty\), we can write
\begin{align*}
  \phi(z) = \frac{z^2}{(z^2 + 9)(z + 2i)^2}
\end{align*}
and
\begin{align*}
  \Res_{z = 2i} f(z) = \frac{\phi'(2i)}{1!} = -\frac{13i}{200}
\end{align*}
So we can write
\begin{align*}
  \int^R_{-R} f(x) dx + \int_{C_R} f(z) dz
  = 2\pi i \left( \frac{3i}{50} - \frac{13i}{200} \right)
  = \frac{\pi}{100}
\end{align*}
For \(z\) on \(C_R\),
\begin{align*}
  \left| \frac{z^2}{(z^2 + 9)(z^2 + 4)^2} \right|
  &= \frac{|z^2|}{|z^2 + 9||z^2 + 4|^2} \\
  &\leq \frac{|z|^2}{(|z|^2 - 9)(|z|^2 - 4)^2}
  = \frac{R^2}{(R^2 - 9)(R^2 - 4)^2}
\end{align*}
so
\begin{align*}
  \left| \int_{C_R} f(z) dz \right| \leq \frac{R^2}{(R^2 - 9)(R^2 - 4)^2} \cdot \pi R = \frac{\pi R^3}{(R^2 - 9)(R^2 - 4)^2} 
\end{align*}
and the contour integral over \(C_R\) tends to zero as \(R \to \infty\).
Since \(f(-z) = f(z)\), we can write
\begin{align*}
  \int^\infty_0 \frac{x^2 dx}{(x^2 + 9)(x^2 + 4)^2}
  &= \lim_{R \to \infty} \int^R_0 \frac{x^2 dx}{(x^2 + 9)(x^2 + 4)^2} \\
  &= \lim_{R \to \infty} \frac{1}{2} \left( \frac{\pi}{100} - \int_{C_R} f(z) dz \right)
  = \frac{\pi}{200}
\end{align*}

\section{Section 86 \#9}
Let \(q(z) = z^3 + 1,\, f(z) = 1 / q(z)\), \(C_R\) be the circular part of the contour given in the problem, and \(L_R\) be the line segment part connecting \(Re^{i2\pi / 3}\) to 0.
\(f(z)\) has isolated singularities at the zeros of \(q(z)\), which are \(\exp(i\pi / 3), -1, \exp(5i\pi / 3)\) and \(f(z)\) is analytic anywhere else.
None of the isolated singularities lies on the real axis.
Only \(\exp(i\pi / 3)\) is inside \(C_R\), so Cauchy's residue theorem gives
\begin{align*}
  \int^R_0 f(x) dx + \int_{C_R} f(z) dz + \int_{L_R} f(z) dz = 2\pi i \Res_{z = \exp(i\pi / 3)} f(z)
\end{align*}
Since \(q(\exp(i\pi / 3)) = 0\) and \(q'(\exp(i\pi / 3)) \not = 0\), we can apply theorem~2 in section~83.
\begin{align*}
  \Res_{z = \exp(i\pi / 3)} f(z) = \frac{1}{q'(\exp(i\pi / 3))} = \frac{1}{3e^{2i\pi / 3}} = \frac{e^{4i\pi / 3}}{3} = -\frac{\sqrt{3}i}{6} - \frac{1}{6}
\end{align*}
so
\begin{align*}
  \int^R_0 f(x) dx + \int_{C_R} f(z) dz + \int_{L_R} f(z) dz
  = 2\pi i \left( -\frac{\sqrt{3}i}{6} - \frac{1}{6} \right)
  = \frac{\sqrt{3}\pi}{3} - \frac{\pi i}{3}
\end{align*}
We can write
\begin{align*}
  \int_{L_R} f(z) dz
  = \int^0_R f(te^{2\pi i / 3}) e^{2\pi i / 3} dt
  = -e^{2\pi i / 3} \int^R_0 \frac{dt}{t^3 \cdot e^{2\pi i} + 1}
  = -e^{2\pi i / 3} \int^R_0 f(t) dt
\end{align*}
Also, for all \(z\) on \(C_R\),
\begin{align*}
  |f(z)| = \left| \frac{1}{z^3 + 1} \right| = \frac{1}{|z^3 + 1|} \leq \frac{1}{|z^3| - 1} = \frac{1}{R^3 - 1}
\end{align*}
so
\begin{align*}
  \left| \int_{C_R} f(z) dz \right| \leq \frac{1}{R^3 - 1} \cdot \frac{2\pi R}{3} = \frac{2\pi R}{3(R^3 - 1)}
\end{align*}
the contour integral over \(C_R\) tends to zero as \(R \to \infty\).
We can write
\begin{align*}
  \int^R_0 f(x) dx + \int_{C_R} f(z) dz + \int_{L_R} f(z) dz
  &= (1 - e^{2\pi i / 3}) \int^R_0 f(x) dx + \int_{C_R} f(z) dz \\
  &= \frac{\sqrt{3}\pi}{3} - \frac{\pi i}{3}
\end{align*}
and get the desired result.
\begin{align*}
  \int^\infty_0 \frac{dx}{x^3 + 1}
  = \lim_{R \to \infty} \int^R_0 \frac{dx}{x^3 + 1}
  = \lim_{R \to \infty} \frac{1}{1 - e^{2\pi i / 3}} \left( \frac{\sqrt{3}\pi}{3} - \frac{\pi i}{3} - \int_{C_R} f(z) dz \right)
  = \frac{2\pi}{3\sqrt{3}}
\end{align*}

\end{document}
