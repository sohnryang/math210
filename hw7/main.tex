\documentclass{scrartcl}
\usepackage[margin=3cm]{geometry}
\usepackage{amsmath}
\usepackage{amssymb}
\usepackage{amsthm}
\usepackage{blindtext}
\usepackage{datetime}
\usepackage{fontspec}
\usepackage{graphicx}
\usepackage{kotex}
\usepackage{mathrsfs}
\usepackage{mathtools}
\usepackage{pgf,tikz,pgfplots}

\pgfplotsset{compat=1.15}
\usetikzlibrary{arrows}

\newcommand\Overline[2][0.8pt]{%
  \begin{tikzpicture}[baseline=(a.base)]
    \node[inner xsep=0pt,inner ysep=1.5pt] (a) {$#2$};
    \draw[line width= #1] (a.north west) -- (a.north east);
  \end{tikzpicture}
}
\newtheorem{theorem}{Theorem}

\setmainhangulfont{Noto Serif CJK KR}[
  UprightFont=* Light, BoldFont=* Bold,
  Script=Hangul, Language=Korean, AutoFakeSlant,
]
\setsanshangulfont{Noto Sans CJK KR}[
  UprightFont=* DemiLight, BoldFont=* Medium,
  Script=Hangul, Language=Korean
]
\setmathhangulfont{Noto Sans CJK KR}[
  SizeFeatures={
    {Size=-6,  Font=* Medium},
    {Size=6-9, Font=*},
    {Size=9-,  Font=* DemiLight},
  },
  Script=Hangul, Language=Korean
]
\title{MATH210: Homework 7 (due Apr. 25)}
\author{손량(20220323)}
\date{Last compiled on: \today, \currenttime}

\newcommand{\un}[1]{\ensuremath{\ \mathrm{#1}}}

\begin{document}
\maketitle

\section{Section 65 \#2}
\subsection{Solution for (a)}
We can write
\begin{align*}
  f(z)
  = \sum^\infty_{n = 0} \frac{f^{(n)}(1)}{n!} (z - 1)^n
  = \sum^\infty_{n = 0} \frac{e}{n!} (z - 1)^n
  = e \sum^\infty_{n = 0} \frac{(z - 1)^n}{n!} \quad (|z - 1| < \infty)
\end{align*}

\subsection{Solution for (b)}
Using Maclaurin series expansion of \(e^z\), we can write
\begin{align*}
  e^z
  = e \cdot e^{z - 1}
  = e \sum^\infty_{n = 0} \frac{(z - 1)^n}{n!} \quad (|z - 1| < \infty)
\end{align*}

\section{Section 65 \#3}
Using the Maclaurin series expansion of \(1 / (1 - z)\), we can write
\begin{align*}
  f(z)
  &= \frac{z}{z^4 + 4}
  = \frac{z}{4} \cdot \frac{1}{1 + (z^4 / 4)}
  = \frac{z}{4} \sum^\infty_{n = 0} \left( -\frac{z^4}{4} \right)^n \\
  &= \frac{z}{4} \sum^\infty_{n = 0} \frac{(-1)^n}{2^{2n}} z^{4n}
  = \sum^\infty_{n = 0} \frac{(-1)^n}{2^{2n + 2}} z^{4n + 1} \quad (|z| < \sqrt{2})
\end{align*}

\section{Section 65 \#4}
Using the provided identity and Maclaurin series expansion of \(\sin z\),
\begin{align*}
  \cos z
  &= -\sin \left( z - \frac{\pi}{2} \right) \\
  &= -\sum^\infty_{n = 0} (-1)^n \frac{(z - \pi / 2)^{2n + 1}}{(2n + 1)!} \quad (|z - \pi / 2| < \infty) \\
  &= \sum^\infty_{n = 0} \frac{(-1)^{n + 1}}{(2n + 1)!} \left( z - \frac{\pi}{2} \right)^{2n + 1} \quad (|z - \pi / 2| < \infty)
\end{align*}

\section{Section 65 \#9}
Using the Maclaurin series expansion of \(\sin z\),
\begin{align}
  \nonumber f(z)
  \nonumber &= \sin (z^2)
  \nonumber = \sum^\infty_{n = 0} (-1)^{n} \frac{z^{2(2n + 1)}}{(2n + 1)!} \\
  \label{sec4_explicit} &= \sum^\infty_{n = 0} (-1)^n \frac{z^{4n + 2}}{(2n + 1)!} \quad (|z^2| < \infty) \\
  \label{sec4_def} &= \sum^\infty_{n = 0} \frac{f^{(n)}(0)}{n!} z^n \quad (|z| < \infty)
\end{align}
From equality of (\ref{sec4_explicit}) and (\ref{sec4_def}), we know that
\begin{align*}
  f^{(n)}(0) = \begin{cases}
    (-1)^k \frac{(4k + 2)!}{(2n + 1)!} & (n = 4k + 2) \\
    0 & (n \not = 4k + 2)
  \end{cases}
\end{align*}
where \(k\) is an integer.
Then, there is no integer \(n, k\) such that \(4n = 4k + 2\) or \(2n + 1 = 4k + 2\), so we get the desired result.
\begin{align*}
  f^{(4n)}(0) = 0, \quad f^{(2n + 1)}(0) = 0
\end{align*}

\section{Section 72 \#1}
Differentiating \(1 / (1 - z)\), we get \(1 / (1 - z)^2\) and term by term differentiation of the given series gives
\begin{align}\label{sec5_diffed}
  \frac{d}{dz} \sum^\infty_{n = 0} z^n = \sum^\infty_{n = 0} (n + 1)z^n \quad (|z| < 1)
\end{align}
and the radius of convergence does not change by theorem~2 in section~71.
Differentiating \(1 / (1 - z)^2\) again, we get \(2 / (1 - z)^3\) and term by term differentiation of (\ref{sec5_diffed}) gives
\begin{align*}
  \frac{d}{dz} \sum^\infty_{n = 0} (n + 1)(n + 2)z^n \quad (|z| < 1)
\end{align*}
again, by theorem~2 in section~71 the radius of convergence does not change.

\section{Section 72 \#3}
Using the Maclaurin series expansion of \(1 / (1 - z)\),
\begin{align}
  \nonumber \frac{1}{z}
  \nonumber &= \frac{1}{2 + (z - 2)}
  \nonumber = \frac{1}{2} \cdot \frac{1}{1 + (z - 2) / 2} \\
  \label{sec6_series} &= \frac{1}{2} \sum^\infty_{n = 0} \left( -\frac{z - 2}{2} \right)^n \quad \left( \left| -\frac{z - 2}{2} \right| < 1 \right)
\end{align}
Differentiating \(1 / z\) yields \(-1 / z^2\), and by differentiating (\ref{sec6_series}), we get
\begin{align*}
  -\frac{1}{z^2}
  = \frac{1}{2} \frac{d}{dz} \sum^\infty_{n = 0} \left( -\frac{z - 2}{2} \right)^n
  &= \frac{1}{2} \frac{d}{dz} \sum^\infty_{n = 0} \left[ \left( -\frac{1}{2} \right)^n (z - 2)^n \right] \\
  &= \frac{1}{2} \sum^\infty_{n = 1} \left[ n \left( -\frac{1}{2} \right)^n (z - 2)^{n - 1} \right] \\
  &= \frac{1}{2} \sum^\infty_{n = 0} \left[ (n + 1) \left( -\frac{1}{2} \right)^{n + 1} (z - 2)^n \right] \\
  &= -\frac{1}{4} \sum^\infty_{n = 0} (-1)^n (n + 1) \left( \frac{z - 2}{2} \right)^n \quad \left( \left| -\frac{z - 2}{2} \right| < 1 \right)
\end{align*}
The radius of convergence does not change after differentiation by theorem~2 in section~71.
Since \(|-(z - 2) / 2| < 1\) is equivalent to \(|z - 2| < 2\), we get the desired result.
\begin{align*}
  \frac{1}{z^2} = \frac{1}{4} \sum^\infty_{n = 0} (-1)^n (n + 1) \left( \frac{z - 2}{2} \right)^n \quad (|z - 2| < 2)
\end{align*}

\section{Section 72 \#6}
We know that \(\operatorname{Log} z\) is an antiderivative of \(1 / z\) on the domain \(D := \{z \in \mathbb{C};\; |z - 1| < 1\}\) since
\begin{align*}
  \frac{d}{dz} \operatorname{Log} z = \frac{1}{z} \quad (|z| > 0, -\pi < \operatorname{Arg} z < \pi)
\end{align*}
holds for all \(z\) such that \(|z - 1| < 1\).
By the theorem in section~48, for all contours lying in \(D\) extending from 1 to \(z\), we can write
\begin{align*}
  \int_C \frac{1}{w} dw
  = \int^z_1 \frac{1}{w} dw
  = \operatorname{Log} z - \operatorname{Log} 1
  = \operatorname{Log} z
\end{align*}
Also, using theorem~1 in section~71, we can write
\begin{align}
  \label{sec7_series} \int_C \sum^\infty_{n = 0} (-1)^n (w - 1)^n dw
  = \sum^\infty_{n = 0} (-1)^n \int_C (w - 1)^n dw \quad (|z - 1| < 1)
\end{align}
for all contour \(C\) interior to the circle \(|z - 1| < 1\).
Since \((w - 1)^n\) has its antiderivative \((n + 1)^{-1} (w - 1)^{n + 1}\) for all integer \(n \geq 0\), for all contour \(C\) in the circle \(|z - 1| < 1\) extending from 1 to \(z\), we can write (\ref{sec7_series}) as
\begin{align*}
  \sum^\infty_{n = 0} (-1)^n \int_C (w - 1)^n dw
  &= \sum^\infty_{n = 0} (-1)^n \frac{(z - 1)^{n + 1}}{n + 1} \\
  &= \sum^\infty_{n = 1} \frac{(-1)^{n + 1}}{n} (z - 1)^n \quad (|z - 1| < 1)
\end{align*}
and get the desired result.
\begin{align}\label{sec7_result}
  \operatorname{Log} z = \sum^\infty_{n = 1} \frac{(-1)^{n + 1}}{n} (z - 1)^n \quad (|z - 1| < 1)
\end{align}

\section{Section 72 \#7}
Using (\ref{sec7_result}), we can write
\begin{align*}
  \frac{\operatorname{Log} z}{z - 1}
  = \sum^\infty_{n = 1} \frac{(-1)^{n + 1}}{n} (z - 1)^{n - 1} = \sum^\infty_{n = 0} \frac{(-1)^n}{n + 1} (z - 1)^n \quad (0 < |z - 1| < 1)
\end{align*}
Since the series on the right hand side evaluates to 1 where \(z = 1\), we can write
\begin{align}\label{sec8_series}
  f(z)
  = \sum^\infty_{n = 1} \frac{(-1)^{n + 1}}{n} (z - 1)^{n - 1} = \sum^\infty_{n = 0} \frac{(-1)^n}{n + 1} (z - 1)^n \quad (|z - 1| < 1)
\end{align}
Using ratio test,
\begin{align*}
  \lim_{n \to \infty} \left| \frac{(-1)^{n + 1}}{n + 2} (z - 1)^{n + 1} \right| \left| \frac{(-1)^n}{n + 1} (z - 1)^n \right|^{-1}
  &= \lim_{n \to \infty} \left| \frac{n + 1}{n + 2} (z - 1) \right| \\
  &= |z - 1| < 1
\end{align*}
so (\ref{sec8_series}) converges absolutely if \(|z - 1| < 1\).
By theorem~1 in section~72, (\ref{sec8_series}) is the Taylor series expansion for \(f\) in powers of \(z - 1\), so \(f\) is analytic throughout a disk \(|z - 1| < 1\).
Also, \(\operatorname{Log} z\) is analytic throughout \(\{z \in \mathbb{C};\; |z| > 0\; \text{and}\; -\pi < \operatorname{Arg} z < \pi\}\) so \(f\) is also analytic throughout \(D := \{z \in \mathbb{C};\; |z| > 0\; \text{and}\; z \not = 1\; \text{and}\; -\pi < \operatorname{Arg} z < \pi\}\).
Since \(f\) is analytic throughout both \(D\) and disk \(|z - 1| < 1\), \(f\) is analytic throughout the domain \(0 < |z| < \infty, -\pi < \operatorname{Arg} z < \pi\).

\end{document}
