\documentclass{scrartcl}
\usepackage[margin=3cm]{geometry}
\usepackage{amsmath}
\usepackage{amssymb}
\usepackage{amsthm}
\usepackage{blindtext}
\usepackage{datetime}
\usepackage{fontspec}
\usepackage{graphicx}
\usepackage{kotex}
\usepackage{mathrsfs}
\usepackage{mathtools}
\usepackage{pgf,tikz,pgfplots}

\pgfplotsset{compat=1.15}
\usetikzlibrary{arrows}

\newcommand\Overline[2][0.8pt]{%
  \begin{tikzpicture}[baseline=(a.base)]
    \node[inner xsep=0pt,inner ysep=1.5pt] (a) {$#2$};
    \draw[line width= #1] (a.north west) -- (a.north east);
  \end{tikzpicture}
}
\newtheorem{theorem}{Theorem}

\setmainhangulfont{Noto Serif CJK KR}[
  UprightFont=* Light, BoldFont=* Bold,
  Script=Hangul, Language=Korean, AutoFakeSlant,
]
\setsanshangulfont{Noto Sans CJK KR}[
  UprightFont=* DemiLight, BoldFont=* Medium,
  Script=Hangul, Language=Korean
]
\setmathhangulfont{Noto Sans CJK KR}[
  SizeFeatures={
    {Size=-6,  Font=* Medium},
    {Size=6-9, Font=*},
    {Size=9-,  Font=* DemiLight},
  },
  Script=Hangul, Language=Korean
]
\title{MATH210 Homework 2 (due Mar. 14)}
\author{손량(20220323)}
\date{Last compiled on: \today, \currenttime}

\newcommand{\un}[1]{\ensuremath{\ \mathrm{#1}}}

\begin{document}
\maketitle

\section{Section 26 \#2 (b)}
Let's check for differentiability.
We can write the component functions as follows:
\[
  u(x, y) := 2xy, \quad v(x, y) := x^2 - y^2
\]

By applying the Cauchy-Riemann equation,
\[
  u_x = v_y \Longrightarrow 2y = -2y, \quad u_y = -v_x \Longrightarrow 2x = -2x
\]

We can know that the function is only differentiable at \(0\), using the sufficient condition for differentiability.
If \(f\) is analytic at \(z_0\), it should be differentiable in some neighborhood of \(z_0\).
Thus, \(f\) cannot be analytic in nonzero \(z_0\) since \(f\) is not differentiable in \(z_0\), let alone a neighborhood of \(z_0\).
For \(z_0 = 0\), there exists some \(z \in D(0, r)\) for all \(r > 0\) where \(f\) is not differentiable at \(z\).
In other words, \(f\) cannot be differentiable in any open set containing \(0\), so \(f\) is not analytic at \(z_0 = 0\).
In conclusion, \(f\) is nowhere analytic.

\section{Section 26 \#6}
For all points in its domain, the component functions \(u(r, \theta) = \ln r, v(r, \theta) = \theta\) have the first-order partial derivatives with respect to \(r\) and \(\theta\).
By applying the polar form of the Cauchy-Riemann equation,
\[
  ru_r = v_\theta \Longrightarrow 1 = 1, \quad u_\theta = -rv_r \Longrightarrow 0 = 0
\]

We can see that the equation holds for every point in the domain.
From the theorem in section 24, it follows that \(f\) is differentiable over its whole domain.
The derivative can be written as the following:
\[
  f'(z) = f'(e^{i\theta}) = e^{-i\theta} (u_r + iv_r) = \frac{1}{r} e^{-i\theta} = \frac{1}{z}
\]

Let \(h(z) = z^2 + 1\).
For all \(z_0\), we can write
\begin{align*}
  h'(z_0) &= \lim_{\Delta z \to 0} \frac{f(z_0 + \Delta z) - f(z_0)}{\Delta z} = \lim_{\Delta z \to 0} \frac{(z + \Delta z)^2 - z^2}{\Delta z} = \lim_{\Delta z \to 0} \frac{2z\Delta z + \Delta z^2}{\Delta z} \\
          &= \lim_{\Delta z \to 0} (2z + \Delta z) = 2z
\end{align*}
and conclude that \(h\) is analytic.
Furthermore, we can observe that
\[
  h(z) = h(x + iy) = (x + iy)^2 + 1 = (x^2 - y^2 + 1) + 2xyi
\]
and \(\operatorname{Im} h(z) = 2xy > 0\) where \(x > 0, y > 0\).
From this, \(r > 0, 0 < \theta < 2\pi\) holds where \(r\exp(i\theta) = h(x + iy)\) and \(x > 0, y > 0\).
This means that \(G(z)\) is defined for all \(z = x + iy\) where \(x > 0, y > 0\) and thus analytic, using the chain rule.
Also, we can know that the following holds in the quadrant \(x > 0, y > 0\).
\[
  G'(z) = \frac{d}{dz} g(h(z)) = g'(h(z)) h'(z) = \frac{2z}{z^2 + 1}
\]

\section{Section 26 \#7}
By the definition of analytic functions, \(f\) is differentiable everywhere in \(D\).
We can write \(f(z) = f(x + iy) = u(x, y) + iv(x, y)\), and \(v(x, y) = 0\) everywhere in \(D\).
By the Cauchy-Riemann equation,
\[
  u_x = v_y = 0, \quad u_y = -v_x = 0
\]

This means that \(f'(z) = u_x(x, y) + iv_x(x, y) = 0\) everywhere in \(D\), so \(f(z)\) is constant throughout \(D\) by the theorem in section 25.

\section{Section 27 \#1}
Using the polar form of the Cauchy-Riemann equation, the following holds.
\[
  ru_r = v_\theta, u_\theta = -rv_r
\]

Since \(f\) is analytic in \(D\), the first-order partial derivatives of \(u\) and \(v\) with respect to \(r\) and \(\theta\) exists in \(D\).
By the assumption of continuity of partial derivatives, we can differentiate both sides of \(ru_r = v_\theta\) with respect to \(r\).
\[
  u_r + ru_{rr} = v_{\theta r}
\]
Differentiating both sides of \(u_\theta = -rv_r\), we get
\[
  u_{\theta\theta} = -rv_{r\theta}
\]
By the continuity of partial derivatives, \(v_{\theta r} = v_{r\theta}\) holds.
We can write
\[
  -rv_{r\theta} = u_{\theta\theta} = -r(u_r + ru_{rr})
\]
Rearranging the equation, we get the desired result.
\[
  r^2 u_{rr}(r, \theta) + ru_r(r, \theta) + u_{\theta\theta}(r, \theta) = 0
\]

\section{Section 27 \#2}
By implicit function theorem, the following holds for \(u\) and \(v\).
\[
  \frac{\partial u}{\partial x} + \frac{\partial u}{\partial y} \frac{dy}{dx} = 0, \quad \frac{\partial v}{\partial x} + \frac{\partial v}{\partial y} \frac{dy}{dx} = 0
\]

Since \(f'(z_0) = f_x(z_0) \not = 0\), \(u(x_0, y_0) \not = 0\) and \(v(x_0, y_0) \not = 0\) holds.
Then the slope of the tangent line of \(u(x, y) = c_1\) at \((x_0, y_0)\) is \(dy/dx = -u_x / u_y\), and the slope of the tangent line of \(v(x, y) = c_2\) at \((x_0, y_0)\) is \(dy/dx = -v_x / v_y\).
By the Cauchy-Riemann equation, \(u_x = v_y, u_y = -v_x\) holds, so \(-v_x / v_y = u_y / u_x\).
The slope of \(u(x, y) = c_1\) is \(-u_x / u_y\) and the slope of \(v(x, y) = c_2\) is \(u_y / u_x\).
Since \((-u_x / u_y)(u_y / u_x) = -1\), the tangent lines are perpendicular.

\section{Section 29 \#1}
Suppose that the \(f(z)\) has a constant value \(w_0\) throughout some neighborhood in \(D\).
A constant function \(f(z) = w_o\) can be such function.
By the theorem in section 28, such \(f\) is unique, so only the constant function \(f(z) = w_0\) can be constant throughout the neighborhood in \(D\), which is contradiction.
Thus, the \(f(z)\) which is analytic and not constant throughout \(D\) cannot be constant throughout any neighborhood contained in \(D\).

\section{Section 29 \#2}
Let \(D_1, D_2, D_3\) be domains of \(f_1, f_2, f_3\), respectively.
Then there exists an open set contained in \(D_1 \cap D_2\), and same holds for \(D_2 \cap D_3\).
Since \(f_1(z) = f_2(z)\) for every point in \(D_1 \cap D_2\), \(f_2\) is an analytic continuation of \(f_1\) by the theorem in section 28.

Likewise, \(f_2(z) = f_3(z)\) for every point in \(D_2 \cap D_3\) implies \(f_3\) is an analytic continuation of \(f_2\) the theorem in section 28.

For \(z = x + iy\) where \(x > 0, y > 0\), \(z\) can be written as \(r\exp(i\theta) = r\exp(i(\theta + 2\pi)) = r(\cos\theta + i\sin\theta)\) where \(r > 0, 0 < \theta < \pi / 2\).
\(f_1(z)\) can be written as the following:
\[
  f_1(z) = f_1(re^{i\theta}) = \sqrt{r}e^{i\theta / 2}
\]
\(f_3(z)\) can be written as the following:
\[
  f_3(z) = f_3(re^{i(\theta + 2\pi)}) = \sqrt{r}e^{i(\theta + 2\pi) / 2} = \sqrt{r}e^{i\theta / 2}e^{i\pi} = -\sqrt{r}e^{i\theta / 2}
\]

Thus, \(f_3(z) = -f_1(z)\) holds.
\end{document}
